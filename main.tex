% !TEX --enable-write18
\documentclass[12pt,a4paper]{report}
%%%%%%%%%%%%%%%%%%%%%%%%%%%%%%%%%%%%%%%%%%%%%%%%%%%%%%%%%%%%%%%
%%%%%%%%%%%%%all_packages%%%%%%%%%%%%%%%%%%%%%%%%%%%%%%%%%%%%%%
\usepackage[table, dvipsnames, svgnames]{xcolor} 
\usepackage[T1]{fontenc}
\usepackage{lmodern} % needed on MikTeX
\usepackage[utf8]{inputenc}
\usepackage{imakeidx}
\usepackage[a4paper,portrait,margin=1in,headheight=15pt]{geometry}
\usepackage{graphicx}
\usepackage{appendix}
\usepackage{caption}
\usepackage{subcaption}
\usepackage{titlesec}
\usepackage{siunitx}
\usepackage[hidelinks,pdfa,pdfusetitle,pageanchor]{hyperref}
\usepackage[toc]{glossaries}
\usepackage{pdfpages,caption}
\usepackage{array}
\usepackage{float}
\usepackage{fancyhdr}
\usepackage{longtable}
\usepackage{siunitx}
\usepackage{enumitem}
\usepackage{ulem}
\usepackage{multirow}
\usepackage{afterpage}
\usepackage{rotating}
\usepackage{acronym}
\usepackage{tfrupee}  
\usepackage{etoolbox}
\usepackage{listings}
\usepackage{hyperref}
\usepackage{textcomp}
\usepackage{tikz}
\usetikzlibrary{automata, positioning}

\usepackage[abspage,user,lastpage]{zref}
\usepackage[hang]{footmisc}       %for aligning footnote

\usepackage[backend=biber,refsection=section,sorting=none,style=numeric, defernumbers=true,safeinputenc]
{biblatex}
\makeindex[columns = 2, title = AlphabeticalIndex, intoc, options =-s indcomma]
%%%%%%%%%%%%%%%%%%%%%%%%%%%%%%%%%%%%%%%%%%%%%%%%%%%%%%%%%%%%%%%
%%%%%%%%%%%%%%%%%%%%%%%%%%%%%%%%%%%%%%%%%%%%%%%%%%%%%%%%%%%%%%%
%%%%%%%%%%%%%%%%%%%preamble and packages for first page%%%%%%%%%%%%%%%%%%%%%%%%%
% \usepackage[dvipsnames, svgnames]{xcolor}
\usepackage{tikz}
\usepackage{xspace}
\usepackage{pdfpages}
\usepackage{afterpage}
\usepackage{graphicx}
\usepackage{hyperxmp}
\usepackage{url}
\urlstyle{rm}
\def\UrlFont{\bfseries\rmfamily\color{blue}} 
\usepackage{ccicons} % use Creative Commons icons
\usepackage[absolute]{textpos} % exact position logo
\usepackage{amsmath} % https://ctan.org/pkg/amsmath for mathematical features
\makeindex[columns = 2, title = Alphabetical Index, intoc, options =-s indcomma]

%% Fonts and typography
%\usepackage{helvet}           
\usepackage[scaled=0.95]{helvet} % Helvetica (kerning adjusted)
\renewcommand{\sfdefault}{phv}     % Helvetica
\usepackage[final]{microtype}  % Improved typography

\DeclareUnicodeCharacter{2060}{}
\definecolor{tuberlingrayarea}{RGB}{249 247 247} % gray #F9F7F7 for background
\definecolor{tuberlindarkgray}{RGB}{91 75 75} % gray #5B4B4B for headings
%%%%%%%%%%%%%%%%%%%%%%%%%%%%%%%%%%%%%%%%%%%%%%%%%%%%%%%%%%%%%%%%%%%%%%%%%%%%%%%%%%%%%%

% to remove the chapter word in the pages
\titleformat{\chapter}[display]
{\normalfont\bfseries}{}{0pt}{\Huge}

% The following definition is copied from authortitle.bbx/authoryear.bbx
\defbibenvironment{nolabelbib}
  {\list
     {}
     {\setlength{\leftmargin}{\bibhang}%
      \setlength{\itemindent}{-\leftmargin}%
      \setlength{\itemsep}{\bibitemsep}%
      \setlength{\parsep}{\bibparsep}}}
  {\endlist}
  {\item}

% adding bibliography sources
\addbibresource{Files/Chapters/Appendices/references.bib}
\nocite{*}

% Page style
\pagestyle{fancy}
\fancyhf{}
\fancyhead[R]{\thepage}
\fancyhead[L]{\nouppercase{\leftmark}}


%footnote style
\setlength\footnotemargin{10pt}
\makenoidxglossaries
\renewcommand{\glsnamefont}[1]{\makefirstuc{#1}}

\newglossaryentry{6/6 vision}{
    name={6/6 vision},
    description={denotes normal eyesight, where an individual can see details from a distance of 6 meters equivalent to what a person with typical vision sees at the same distance}
}
\newglossaryentry{24-hour clock}{
    name=24-hour clock,
    description={It is a timekeeping system where the hours beyond 12:00 are expressed without the use of AM or PM indicators.}
}
\newglossaryentry{accelerometer}{
    name=accelerometer,
    description={An instrument for measuring the acceleration of a body, which in practical terms means changes in speed or direction of a motion.}
}
\newglossaryentry{audio unit}{
    name={audio unit},
    description={Provides audio announcements using speakers regarding the estimated arrival time of the bus at specific bus stops.}
}
\newglossaryentry{DC/DC converter}{
    name=DC/DC converter,
    description={A DC-DC converter is an electronic circuit or electromechanical device that converts a source of direct current from one voltage level to another.}
}
\newglossaryentry{embedded system}{
    name=Embedded system,
    description={An embedded system is a specialized computer system designed to perform specific tasks within a larger system or device.}
}
\newglossaryentry{GPS}{
    name=GPS,
    description={GPS stands for Global Positioning System. It is a satellite-based navigation system that enables users to determine their precise location anywhere on Earth.}
}
\newglossaryentry{LED dot matrix display}{
    name=LED dot matrix display,
    description={A grid of light-emitting diodes (LEDs) arranged in rows and columns, allowing each LED to be individually controlled to display characters by selectively turning on or off specific LEDs in the matrix.}
}
\newglossaryentry{MPPT converter}{
    name=MPPT converter,
    description={Stands for  Maximum Power Point Tracking , it is a type of DC/DC converter commonly used in photovoltaic systems to optimize the power output of solar panels by continuously adjusting the operating point to maximize energy harvest from the panels.}
}
\newglossaryentry{PWM converter}{
    name=PWM converter,
    description={A type of power electronics device used to regulate the output voltage or current by controlling the duty cycle of a pulse-width modulated signal.}
}
\newglossaryentry{round trip}{
    name=round trip,
    description={Refers to a journey where a bus travels from a starting point to a destination and then returns back to the original starting point.}
}
\newglossaryentry{smog}{
    name=smog,
    description={A type of air pollution that consists of a mixture of smoke and fog, often caused by industrial emissions, vehicle exhaust, and atmospheric conditions.}
}
\newglossaryentry{sliding text animation}{
    name=sliding text animation,
    description={It is a visual effect over a display where text moves smoothly across the screen horizontally, vertically, or diagonally.}
}
\newglossaryentry{stall period}{
    name=stall period,
    description={Refers to a duration of time during which the system pauses temporarily until the situation is resolved, allowing the system to resume its intended function.}
}
\newglossaryentry{USB}{
    name=USB,
    description={Stands for Universal Serial Bus. It is a standardized interface commonly used for connecting peripherals to computers and other electronic devices.}
}
\newglossaryentry{ultrasonic sensor}{
    name=ultrasonic sensor,
    description={A device that utilizes ultrasonic waves, typically operating in the frequency range of 20 kHz to 200 kHz, to detect the presence and distance objects by emitting ultrasonic pulses and measuring the time it takes for the pulses to reflect back from objects.}
}
\newglossaryentry{U-turn}{
    name=U-turn,
    description={When a vehicle turns 180 degrees to reverse its direction and proceed in the opposite direction.}
}
\newglossaryentry{Wi-Fi}{
    name=Wi-Fi,
    description={Wireless Fidelity, is a technology that allows electronic devices to connect to the internet or communicate with one another wirelessly using radio frequencies.}
}


%defining new alignment
\newcolumntype{n}{>{\centering} m{5.5cm}}
\newcolumntype{e}{>{\raggedleft} m{3.1cm}}
\newcolumntype{E}{>{\raggedleft} m{5.1cm}}


%%%%%%%%%%%%%%%%%%%%%%%%%%%%%%%%%%%%%%%%%%%%%%%%%%%%%%%%%%%%%%%%%
%%%             Landscape Environment                         %%%
%%%        (Must load etoolbox package before)                %%%
%%%    And use /savegeometry before the using landscape page  %%%
%%%      and use /loadgeometry after the landscape page       %%%
%%%%%%%%%%%%%%%%%%%%%%%%%%%%%%%%%%%%%%%%%%%%%%%%%%%%%%%%%%%%%%%%%

\makeatletter
 \def\ifGm@preamble#1{\@firstofone}
  \appto\restoregeometry{%
     \pdfpagewidth=\paperwidth
     \pdfpageheight=\paperheight}
 \apptocmd\newgeometry{%
    \pdfpagewidth=\paperwidth
    \pdfpageheight=\paperheight}{}{}
 \makeatother


\newenvironment{landscapes}{
    \newpage
    \newgeometry{margin=1in,landscape}  
}{%
\newpage
}

\fancypagestyle{uselscape}{
	\fancyhf{}
	\fancyhead[R]{\thepage}
	\fancyhead[L]{\nouppercase{\rightmark}}
	\fancyheadoffset[R]{0mm}
	}

%%%%%%%%%%%%%%%%%%%%%%%% Start of Document %%%%%%%%%%%%%%%%%%%%

\begin{document}
\renewcommand{\thefootnote}{\alph{footnote}}
\def\mydate{\leavevmode\hbox{\the\year-\twodigits\month-\twodigits\day}}
\def\twodigits#1{\ifnum#1<10 0\fi\the#1}
\begin{titlepage}
    \newgeometry{voffset=-1in,textwidth=160mm, textheight=253mm,margin=1in}
   % \newgeometry{hoffset=-1in, voffset=-1in, textwidth=160mm, textheight=253mm, topmargin=22mm, headsep=0mm, headheight=0mm, lmargin=2.2in, marginparpush=0mm,footheight=0mm, footsep=0mm}

    % Logo right corner - change to black-white if needed
       \begin{textblock*}{2cm}(14cm,4cm)
       \includegraphics[width=2in]{Files/Images/iitdlogo.png}
       %\includegraphics[width=50mm]{def/TU_Logo_lang_RGB_schwarz.pdf}
       \end{textblock*}
         %%% gray background
            \vskip1in
            \begin{tikzpicture}[remember picture, overlay]         
	        \fill [tuberlingrayarea, opacity=1] (-1.2,-7.0) rectangle (16,-26);
	        \end{tikzpicture}
	        \raggedright
            \vskip3in
             %%% title elements
              {%
				\fontsize{20}{28}
				\boldmath
				\sffamily
				 Tribe: CleanTeX
				\par
            }
             \vskip0.2in
           
             {
				\Large
				\fontsize{28}{32}
				\bfseries
				\boldmath
				\sffamily
                Bus Stand Display Unit
				\par
            }
            \vskip0.1in
             {
				\fontsize{18}{18}
				\sffamily
                
				\par
            }
            \vskip0.1in
             {
				\fontsize{16}{16}
				\sffamily
                ELP305 : Design and System Laboratory
				\par
            }
            \vskip0.1in
             {
				\fontsize{16}{14}
				\sffamily
                \textbf{Indian Institute of Technology, Delhi}
                \par
            }
    \vspace{0.85in}
    {
        \normalsize
        %\fontsize{10}{12}
        \sffamily
        \textbf{\textcolor{tuberlindarkgray}{Document ID}}\par \vspace{0.12cm}
                \hyperlink{Target_DocInfo}{\textbf{P2RS-\mydate-\input{ver}}}\par
                \vspace{0.24cm}  
        \textbf{\textcolor{tuberlindarkgray}{Document Authorised by}}\par \vspace{0.12cm}
        {\href{https://www.linkedin.com/in/ayush-gupta-undergraduate/}{Ayush Gupta}, \href{http://linkedin.com/in/aryan-mishra-04j}{Aryan Mishra}}

        \vspace{0.24cm}
        \textbf{\textcolor{tuberlindarkgray}{Release Date}}\par \vspace{0.12cm}
        {\today}

        \vspace{0.24cm}
        \textbf{\textcolor{tuberlindarkgray}{Release Version}}\par \vspace{0.12cm}
        \textbf{\hyperlink{Target_DocInfo}{\input{ver}}}\par
    \vspace{0.24cm}
        
        \textbf{\textcolor{tuberlindarkgray}{Release Type}}\par \vspace{0.12cm}
        {Private Release}

        \vspace{0.24cm}
        \textbf{\textcolor{tuberlindarkgray}{Terms Of Use}}\par \vspace{0.12cm}
        {All Rights Reserved}

        \vspace{0.24cm}
        \textbf{\textcolor{tuberlindarkgray}{Project Details}}\par \vspace{0.12cm}
        {This project is being managed on \href{https://github.com/ELP305-Cleaning-Machine}{GitHub} and the latest version of this report is hosted \href{https://2nav.github.io/TribeC/}{\textcolor{blue}{here}}}
    }
\end{titlepage}
\afterpage{\restoregeometry}
\pagenumbering{roman}
\tableofcontents
\pagestyle{fancy}
\chapter{Introduction}

\section{Overall Coordinators}
    \begin{center}
	    \label{table:oc}
	    \begin{longtable}{ | n | e | E | c| }
		    \hline
		    \textbf{Name}                                                                & \textbf{Entry Number} & \textbf{Email ID}                                                    & \textbf{IF} \\
		    \hline \hline\href{http://linkedin.com/in/aryan-mishra-04j}{Aryan Mishra} & 2021EE10137 & \href{mailto:ee1210137@ee.iitd.ac.in}{ee1210137@ee.iitd.ac.in} & 1.0\\ 
\hline 
\href{https://www.linkedin.com/in/ayush-gupta-undergraduate/}{Ayush Gupta} & 2021MT10697 & \href{mailto:mt1210697@maths.iitd.ac.in}{mt1210697@maths.iitd.ac.in} & 1.0\\ 
\hline 

		    \caption{Overall Coordinators}
	    \end{longtable}
    \end{center}
    \section[Subtribes]{Subtribes\footnote{\textbf{\textit{Activity Coordinator}} written in bold}}

    \subsection{Documentation}
    \begin{center}
    \label{table:Docu1}
    \begin{longtable}{| n | e | E | c| }
        \hline
        \textbf{Name}                                                                                                      & \textbf{Entry Number} & \textbf{Email ID}                                                    & \textbf{IF} \\
        \hline \hline\href{https://its-archisman.github.io/My-Website/}{\textbf{Archisman Biswas} } & 2021MT10254 & \href{mailto:mt1210254@maths.iitd.ac.in}{mt1210254@maths.iitd.ac.in} & 1.0\\ 
\hline 
\href{https://github.com/Azm1t}{Asmit Singh} & 2021MT10887 & \href{mailto:mt1210887@maths.iitd.ac.in}{mt1210887@maths.iitd.ac.in} & 1.0\\ 
\hline 
\href{https://github.com/Musaibgani}{Musaib Gani Pirzada} & 2021MT10227 & \href{mailto:mt1210227@maths.iitd.ac.in}{mt1210227@maths.iitd.ac.in} & 0.98\\ 
\hline 
\href{github.com/2nav}{Navneet Raj} & 2021MT10240 & \href{mailto:mt1210240@maths.iitd.ac.in}{mt1210240@maths.iitd.ac.in} & 1.0\\ 
\hline 
\href{Alice-Mina}{Neelam Kumari Meena} & 2021MT10938 & \href{mailto:mt1210938@maths.iitd.ac.in}{mt1210938@maths.iitd.ac.in} & 0.95\\ 
\hline 
\href{nan}{Nikhil Choudhary} & 2020MT10826 & \href{mailto:mt1200826@maths.iitd.ac.in}{mt1200826@maths.iitd.ac.in} & 0.9\\ 
\hline 
\href{https://www.linkedin.com/in/purushottam-malviya-9225681bb/}{Purushottam Malviya} & 2020EE10531 & \href{mailto:ee1200531@ee.iitd.ac.in}{ee1200531@ee.iitd.ac.in} & 0.95\\ 
\hline 
\href{https://www.linkedin.com/in/rahul-bhardwaj-dintyala-244117202/}{Rahul Bhardwaj} & 2020MT60877 & \href{mailto:mt6200877@maths.iitd.ac.in}{mt6200877@maths.iitd.ac.in} & 1.0\\ 
\hline 
\href{nan}{Rishabh Barola} & 2021EE10636 & \href{mailto:ee1210636@ee.iitd.ac.in}{ee1210636@ee.iitd.ac.in} & 1.0\\ 
\hline 
\href{nan}{Saket Kandoi} & 2021MT60265 & \href{mailto:mt6210265@maths.iitd.ac.in}{mt6210265@maths.iitd.ac.in} & 0.9\\ 
\hline 
\href{https://github.com/Sanjay23Pooniya}{Sanjay Pooniya} & 2021EE10148 & \href{mailto:ee1210148@ee.iitd.ac.in}{ee1210148@ee.iitd.ac.in} & 1.0\\ 
\hline 
\href{https://www.linkedin.com/in/sanskriti-gautam-1161b6236/}{Sanskriti Gautam} & 2021MT10935 & \href{mailto:mt1210935@maths.iitd.ac.in}{mt1210935@maths.iitd.ac.in} & 1.0\\ 
\hline 
\href{https://iamsecretlyflash.github.io/}{Vaibhav Seth} & 2021MT10236 & \href{mailto:mt1210236@maths.iitd.ac.in}{mt1210236@maths.iitd.ac.in} & 0.95\\ 
\hline 
\href{https://github.com/crownCTDM}{Vasu Sharma} & 2021EE10620 & \href{mailto:ee1210620@ee.iitd.ac.in}{ee1210620@ee.iitd.ac.in} & 0.85\\ 
\hline 
\hline
		    \caption{Documentation}
	    \end{longtable}
    \end{center}
    \subsection{Others}
    \begin{center}
    \label{table:Othe1}
    \begin{longtable}{| n | e | E | c| }
        \hline
        \textbf{Name}                                                                                                      & \textbf{Entry Number} & \textbf{Email ID}                                                    & \textbf{IF} \\
        \hline \hline\href{https://github.com/Hello-3585}{Abhinav Verma} & 2021EE10978 & \href{mailto:ee1210978@ee.iitd.ac.in}{ee1210978@ee.iitd.ac.in} & 1.0\\ 
\hline 
\href{4-tohchalega}{Aditya Jain} & 2021EE10633 & \href{mailto:ee1210633@ee.iitd.ac.in}{ee1210633@ee.iitd.ac.in} & 0.68\\ 
\hline 
\href{https://github.com/Ameya-Mishra}{Ameya Mishra} & 2021MT10637 & \href{mailto:mt1210637@maths.iitd.ac.in}{mt1210637@maths.iitd.ac.in} & 0.68\\ 
\hline 
\href{https://www.linkedin.com/in/anchal-popli-182047225/}{Anchal} & 2021MT10910 & \href{mailto:mt1210910@maths.iitd.ac.in}{mt1210910@maths.iitd.ac.in} & 0.86\\ 
\hline 
\href{https://www.linkedin.com/in/aniket-abhiraj-357381237/}{Aniket Abhiraj} & 2021EE10676 & \href{mailto:ee1210676@ee.iitd.ac.in}{ee1210676@ee.iitd.ac.in} & 0.98\\ 
\hline 
\href{lunatic04}{Aniket Singh} & 2021MT10256 & \href{mailto:mt1210256@maths.iitd.ac.in}{mt1210256@maths.iitd.ac.in} & 0.68\\ 
\hline 
\href{https://github.com/Anshulydav}{Anshul} & 2021EE10729 & \href{mailto:ee1210729@ee.iitd.ac.in}{ee1210729@ee.iitd.ac.in} & 1.0\\ 
\hline 
\href{nan}{Ark Verma} & 2021EE10783 & \href{mailto:ee1210783@ee.iitd.ac.in}{ee1210783@ee.iitd.ac.in} & 0.88\\ 
\hline 
\href{https://github.com/ArnavGoel458}{Arnav Goel} & 2021EE10699 & \href{mailto:ee1210699@ee.iitd.ac.in}{ee1210699@ee.iitd.ac.in} & 0.68\\ 
\hline 
\href{https://www.linkedin.com/in/aryan-gupta-43b283229}{Aryan Gupta} & 2021EE10974 & \href{mailto:ee1210974@ee.iitd.ac.in}{ee1210974@ee.iitd.ac.in} & 1.0\\ 
\hline 
\href{https://github.com/Dhruv-Kushwaha2010}{Dhruv Kushwaha} & 2021MT10235 & \href{mailto:mt1210235@maths.iitd.ac.in}{mt1210235@maths.iitd.ac.in} & 0.88\\ 
\hline 
\href{https://github.com/Harsh2718}{Harsh Agarwal} & 2021EE30977 & \href{mailto:ee3210977@ee.iitd.ac.in}{ee3210977@ee.iitd.ac.in} & 1.0\\ 
\hline 
\href{https://github.com/harshswaika}{Harsh Swaika} & 2021EE11052 & \href{mailto:ee1211052@ee.iitd.ac.in}{ee1211052@ee.iitd.ac.in} & 0.7\\ 
\hline 
\href{https://github.com/HarshitSachdeva03}{Harshit Sachdeva} & 2021EE30705 & \href{mailto:ee3210705@ee.iitd.ac.in}{ee3210705@ee.iitd.ac.in} & 0.68\\ 
\hline 
\href{https://github.com/wm0395/}{Harshit Singh} & 2021MT10257 & \href{mailto:mt1210257@maths.iitd.ac.in}{mt1210257@maths.iitd.ac.in} & 0.86\\ 
\hline 
\href{nan}{Kaustubh Dev} & 2021EE10689 & \href{mailto:ee1210689@ee.iitd.ac.in}{ee1210689@ee.iitd.ac.in} & 0.92\\ 
\hline 
\href{https://www.linkedin.com/in/khushika-shringi-205419226}{Khushika Shringi} & 2021EE10665 & \href{mailto:ee1210665@ee.iitd.ac.in}{ee1210665@ee.iitd.ac.in} & 1.0\\ 
\hline 
\href{https://www.linkedin.com/in/khushvind-maurya/}{Khushvind Maurya} & 2021MT10238 & \href{mailto:mt1210238@maths.iitd.ac.in}{mt1210238@maths.iitd.ac.in} & 0.68\\ 
\hline 
\href{https://github.com/Kinjal001}{Kinjal Anchhara} & 2021MT60959 & \href{mailto:mt6210959@maths.iitd.ac.in}{mt6210959@maths.iitd.ac.in} & 1.0\\ 
\hline 
\href{https://github.com/maithilij2003}{Maithili Joshi} & 2021EE10653 & \href{mailto:ee1210653@ee.iitd.ac.in}{ee1210653@ee.iitd.ac.in} & 0.98\\ 
\hline 
\href{https://github.com/Mohitraj227}{Mohit Raj Modi} & 2021MT10919 & \href{mailto:mt1210919@maths.iitd.ac.in}{mt1210919@maths.iitd.ac.in} & 0.85\\ 
\hline 
\href{https://www.linkedin.com/in/mridulahi/}{Mridul Ahi} & 2021MT10901 & \href{mailto:mt1210901@maths.iitd.ac.in}{mt1210901@maths.iitd.ac.in} & 0.7\\ 
\hline 
\href{nan}{Namay Bedi Verma} & 2021MT61051 & \href{mailto:mt6211051@maths.iitd.ac.in}{mt6211051@maths.iitd.ac.in} & 0.83\\ 
\hline 
\href{oshink}{Oshin Kavdia} & 2021EE10654 & \href{mailto:ee1210654@ee.iitd.ac.in}{ee1210654@ee.iitd.ac.in} & 0.7\\ 
\hline 
\href{https://www.linkedin.com/in/rohandas1710/}{Rohan Das} & 2021EE10621 & \href{mailto:ee1210621@ee.iitd.ac.in}{ee1210621@ee.iitd.ac.in} & 0.98\\ 
\hline 
\href{nan}{Sai Raj Kolisetti} & 2021EE10145 & \href{mailto:ee1210145@ee.iitd.ac.in}{ee1210145@ee.iitd.ac.in} & 1.0\\ 
\hline 
\href{https://www.linkedin.com/in/sarthak-kumar-singh-a77146245/}{Sarthak Kumar Singh} & 2021EE10673 & \href{mailto:ee1210673@ee.iitd.ac.in}{ee1210673@ee.iitd.ac.in} & 0.98\\ 
\hline 
\href{https://in.linkedin.com/in/shubh-chhabra-007197227}{Shubh Chhabra} & 2021EE10645 & \href{mailto:ee1210645@ee.iitd.ac.in}{ee1210645@ee.iitd.ac.in} & 1.0\\ 
\hline 
\href{aggarwalshubham009}{Shubham Aggarwal} & 2021EE10809 & \href{mailto:ee1210809@ee.iitd.ac.in}{ee1210809@ee.iitd.ac.in} & 1.0\\ 
\hline 
\href{https://www.linkedin.com/in/ujjwal-yadav-880448223}{Ujjwal Yadav} & 2021EE10669 & \href{mailto:ee1210669@ee.iitd.ac.in}{ee1210669@ee.iitd.ac.in} & 0.98\\ 
\hline 
\href{Vishal-495}{Vishal Sai Bingi} & 2021EE10668 & \href{mailto:ee1210668@ee.iitd.ac.in}{ee1210668@ee.iitd.ac.in} & 0.7\\ 
\hline 
\href{https://www.linkedin.com/in/yash-goel-6ba26322}{Yash Goel} & 2021EE10984 & \href{mailto:ee1210984@ee.iitd.ac.in}{ee1210984@ee.iitd.ac.in} & 1.0\\ 
\hline 
\href{https://www.linkedin.com/in/abhilasa-das-194413236}{Abhilasa Das} & 2021EE10168 & \href{mailto:ee1210168@ee.iitd.ac.in}{ee1210168@ee.iitd.ac.in} & 1.0\\ 
\hline 
\href{https://github.com/Aditi188}{Aditi Shekhar} & 2021EE10685 & \href{mailto:ee1210685@ee.iitd.ac.in}{ee1210685@ee.iitd.ac.in} & 1.0\\ 
\hline 
\href{https://www.linkedin.com/in/advait-ninawe-7a790022}{Advait Rajesh Ninawe} & 2021EE30714 & \href{mailto:ee3210714@ee.iitd.ac.in}{ee3210714@ee.iitd.ac.in} & 1.0\\ 
\hline 
\href{https://github.com/AyushKumar284}{Ayush Kumar} & 2021EE10150 & \href{mailto:ee1210150@ee.iitd.ac.in}{ee1210150@ee.iitd.ac.in} & 1.0\\ 
\hline 
\href{https://www.linkedin.com/in/chetan-chaurasia-561b3b228}{Chetan Chaurasia} & 2021EE10147 & \href{mailto:ee1210147@ee.iitd.ac.in}{ee1210147@ee.iitd.ac.in} & 0.7\\ 
\hline 
\href{nan}{Chirag Gautam} & 2021EE10166 & \href{mailto:ee1210166@ee.iitd.ac.in}{ee1210166@ee.iitd.ac.in} & 0.7\\ 
\hline 
\href{nan}{Deepanshu Kumar} & 2021EE10696 & \href{mailto:ee1210696@ee.iitd.ac.in}{ee1210696@ee.iitd.ac.in} & 0.4\\ 
\hline 
\href{lost-strings}{Disha Katia} & 2021EE10647 & \href{mailto:ee1210647@ee.iitd.ac.in}{ee1210647@ee.iitd.ac.in} & 1.0\\ 
\hline 
\href{dn09-create}{Durgesh Nandini} & 2021EE10651 & \href{mailto:ee1210651@ee.iitd.ac.in}{ee1210651@ee.iitd.ac.in} & 1.0\\ 
\hline 
\href{https://github.com/vulpeex}{Eepsita} & 2021EE10692 & \href{mailto:ee1210692@ee.iitd.ac.in}{ee1210692@ee.iitd.ac.in} & 1.0\\ 
\hline 
\href{nan}{Garvit Dhoot} & 2021EE30823 & \href{mailto:ee3210823@ee.iitd.ac.in}{ee3210823@ee.iitd.ac.in} & 0.83\\ 
\hline 
\href{https://www.linkedin.com/in/himanshu-prajapati-400669217}{Himanshu} & 2021EE30177 & \href{mailto:ee3210177@ee.iitd.ac.in}{ee3210177@ee.iitd.ac.in} & 0.7\\ 
\hline 
\href{https://github.com/joelAKY}{Joel Arun Kumar Yenubotula} & 2021EE10159 & \href{mailto:ee1210159@ee.iitd.ac.in}{ee1210159@ee.iitd.ac.in} & 1.0\\ 
\hline 
\href{https://github.com/kalu693}{Kalu Ram Tard} & 2021EE10680 & \href{mailto:ee1210680@ee.iitd.ac.in}{ee1210680@ee.iitd.ac.in} & 0.7\\ 
\hline 
\href{nan}{Kartavya Khurana} & 2021EE30710 & \href{mailto:ee3210710@ee.iitd.ac.in}{ee3210710@ee.iitd.ac.in} & 0.98\\ 
\hline 
\href{https://www.linkedin.com/in/pooja-mahajan-101b63227}{Pooja Mahajan} & 2021EE10652 & \href{mailto:ee1210652@ee.iitd.ac.in}{ee1210652@ee.iitd.ac.in} & 1.0\\ 
\hline 
\href{BoredApe07}{Pramukh Jain} & 2021EE10720 & \href{mailto:ee1210720@ee.iitd.ac.in}{ee1210720@ee.iitd.ac.in} & 0.98\\ 
\hline 
\href{https://github.com/raviparihar0659}{Ravi Parihar} & 2021EE10156 & \href{mailto:ee1210156@ee.iitd.ac.in}{ee1210156@ee.iitd.ac.in} & 0.82\\ 
\hline 
\href{https://www.linkedin.com/in/sheetal-manatawal-50119a236}{Sheetal Manatawal} & 2021EE10174 & \href{mailto:ee1210174@ee.iitd.ac.in}{ee1210174@ee.iitd.ac.in} & 0.7\\ 
\hline 
\href{https://github.com/shivam-kumar04}{Shivam Kumar} & 2021EE10165 & \href{mailto:ee1210165@ee.iitd.ac.in}{ee1210165@ee.iitd.ac.in} & 0.7\\ 
\hline 
\href{nan}{Sudhanshu Raj} & 2021EE10132 & \href{mailto:ee1210132@ee.iitd.ac.in}{ee1210132@ee.iitd.ac.in} & 0.0\\ 
\hline 
\href{https://github.com/tanmaimerugu}{Tanmai Merugu} & 2021EE10149 & \href{mailto:ee1210149@ee.iitd.ac.in}{ee1210149@ee.iitd.ac.in} & 0.7\\ 
\hline 
\href{https://github.com/vikas4vikas}{Vikas Meena} & 2021EE10169 & \href{mailto:ee1210169@ee.iitd.ac.in}{ee1210169@ee.iitd.ac.in} & 0.7\\ 
\hline 
\href{nan}{Vinay Sah} & 2021EE10171 & \href{mailto:ee1210171@ee.iitd.ac.in}{ee1210171@ee.iitd.ac.in} & 0.7\\ 
\hline 
\href{https://www.linkedin.com/in/yash089610/}{Yash Agarwal} & 2021EE10638 & \href{mailto:ee1210638@ee.iitd.ac.in}{ee1210638@ee.iitd.ac.in} & 1.0\\ 
\hline 
\hline
		    \caption{Others}
	    \end{longtable}
    \end{center}
\section{Tribe Members with IF less than 1}
    \subsection{Documentation}
    \begin{center}
    \label{Docu2}
    \begin{longtable}{| n | e | E | c| }
        \hline
        \textbf{Name}                                                                                                      & \textbf{Entry Number} & \textbf{Email ID}                                                    & \textbf{IF} \\
        \hline \hline\href{https://github.com/Musaibgani}{Musaib Gani Pirzada} & 2021MT10227 & \href{mailto:mt1210227@maths.iitd.ac.in}{mt1210227@maths.iitd.ac.in} & 0.98\\ 
\hline 
\href{Alice-Mina}{Neelam Kumari Meena} & 2021MT10938 & \href{mailto:mt1210938@maths.iitd.ac.in}{mt1210938@maths.iitd.ac.in} & 0.95\\ 
\hline 
\href{nan}{Nikhil Choudhary} & 2020MT10826 & \href{mailto:mt1200826@maths.iitd.ac.in}{mt1200826@maths.iitd.ac.in} & 0.9\\ 
\hline 
\href{https://www.linkedin.com/in/purushottam-malviya-9225681bb/}{Purushottam Malviya} & 2020EE10531 & \href{mailto:ee1200531@ee.iitd.ac.in}{ee1200531@ee.iitd.ac.in} & 0.95\\ 
\hline 
\href{nan}{Saket Kandoi} & 2021MT60265 & \href{mailto:mt6210265@maths.iitd.ac.in}{mt6210265@maths.iitd.ac.in} & 0.9\\ 
\hline 
\href{https://iamsecretlyflash.github.io/}{Vaibhav Seth} & 2021MT10236 & \href{mailto:mt1210236@maths.iitd.ac.in}{mt1210236@maths.iitd.ac.in} & 0.95\\ 
\hline 
\href{https://github.com/crownCTDM}{Vasu Sharma} & 2021EE10620 & \href{mailto:ee1210620@ee.iitd.ac.in}{ee1210620@ee.iitd.ac.in} & 0.85\\ 
\hline 
\hline
		    \caption{IF less than 1 (Documentation)}
	    \end{longtable}
    \end{center}

    \subsection{Others}
    \begin{center}
    \label{Othe2}
    \begin{longtable}{| n | e | E | c| }
        \hline
        \textbf{Name}                                                                                                      & \textbf{Entry Number} & \textbf{Email ID}                                                    & \textbf{IF} \\
        \hline \hline\href{4-tohchalega}{Aditya Jain} & 2021EE10633 & \href{mailto:ee1210633@ee.iitd.ac.in}{ee1210633@ee.iitd.ac.in} & 0.68\\ 
\hline 
\href{https://github.com/Ameya-Mishra}{Ameya Mishra} & 2021MT10637 & \href{mailto:mt1210637@maths.iitd.ac.in}{mt1210637@maths.iitd.ac.in} & 0.68\\ 
\hline 
\href{https://www.linkedin.com/in/anchal-popli-182047225/}{Anchal} & 2021MT10910 & \href{mailto:mt1210910@maths.iitd.ac.in}{mt1210910@maths.iitd.ac.in} & 0.86\\ 
\hline 
\href{https://www.linkedin.com/in/aniket-abhiraj-357381237/}{Aniket Abhiraj} & 2021EE10676 & \href{mailto:ee1210676@ee.iitd.ac.in}{ee1210676@ee.iitd.ac.in} & 0.98\\ 
\hline 
\href{lunatic04}{Aniket Singh} & 2021MT10256 & \href{mailto:mt1210256@maths.iitd.ac.in}{mt1210256@maths.iitd.ac.in} & 0.68\\ 
\hline 
\href{nan}{Ark Verma} & 2021EE10783 & \href{mailto:ee1210783@ee.iitd.ac.in}{ee1210783@ee.iitd.ac.in} & 0.88\\ 
\hline 
\href{https://github.com/ArnavGoel458}{Arnav Goel} & 2021EE10699 & \href{mailto:ee1210699@ee.iitd.ac.in}{ee1210699@ee.iitd.ac.in} & 0.68\\ 
\hline 
\href{https://github.com/Dhruv-Kushwaha2010}{Dhruv Kushwaha} & 2021MT10235 & \href{mailto:mt1210235@maths.iitd.ac.in}{mt1210235@maths.iitd.ac.in} & 0.88\\ 
\hline 
\href{https://github.com/harshswaika}{Harsh Swaika} & 2021EE11052 & \href{mailto:ee1211052@ee.iitd.ac.in}{ee1211052@ee.iitd.ac.in} & 0.7\\ 
\hline 
\href{https://github.com/HarshitSachdeva03}{Harshit Sachdeva} & 2021EE30705 & \href{mailto:ee3210705@ee.iitd.ac.in}{ee3210705@ee.iitd.ac.in} & 0.68\\ 
\hline 
\href{https://github.com/wm0395/}{Harshit Singh} & 2021MT10257 & \href{mailto:mt1210257@maths.iitd.ac.in}{mt1210257@maths.iitd.ac.in} & 0.86\\ 
\hline 
\href{nan}{Kaustubh Dev} & 2021EE10689 & \href{mailto:ee1210689@ee.iitd.ac.in}{ee1210689@ee.iitd.ac.in} & 0.92\\ 
\hline 
\href{https://www.linkedin.com/in/khushvind-maurya/}{Khushvind Maurya} & 2021MT10238 & \href{mailto:mt1210238@maths.iitd.ac.in}{mt1210238@maths.iitd.ac.in} & 0.68\\ 
\hline 
\href{https://github.com/maithilij2003}{Maithili Joshi} & 2021EE10653 & \href{mailto:ee1210653@ee.iitd.ac.in}{ee1210653@ee.iitd.ac.in} & 0.98\\ 
\hline 
\href{https://github.com/Mohitraj227}{Mohit Raj Modi} & 2021MT10919 & \href{mailto:mt1210919@maths.iitd.ac.in}{mt1210919@maths.iitd.ac.in} & 0.85\\ 
\hline 
\href{https://www.linkedin.com/in/mridulahi/}{Mridul Ahi} & 2021MT10901 & \href{mailto:mt1210901@maths.iitd.ac.in}{mt1210901@maths.iitd.ac.in} & 0.7\\ 
\hline 
\href{nan}{Namay Bedi Verma} & 2021MT61051 & \href{mailto:mt6211051@maths.iitd.ac.in}{mt6211051@maths.iitd.ac.in} & 0.83\\ 
\hline 
\href{oshink}{Oshin Kavdia} & 2021EE10654 & \href{mailto:ee1210654@ee.iitd.ac.in}{ee1210654@ee.iitd.ac.in} & 0.7\\ 
\hline 
\href{https://www.linkedin.com/in/rohandas1710/}{Rohan Das} & 2021EE10621 & \href{mailto:ee1210621@ee.iitd.ac.in}{ee1210621@ee.iitd.ac.in} & 0.98\\ 
\hline 
\href{https://www.linkedin.com/in/sarthak-kumar-singh-a77146245/}{Sarthak Kumar Singh} & 2021EE10673 & \href{mailto:ee1210673@ee.iitd.ac.in}{ee1210673@ee.iitd.ac.in} & 0.98\\ 
\hline 
\href{https://www.linkedin.com/in/ujjwal-yadav-880448223}{Ujjwal Yadav} & 2021EE10669 & \href{mailto:ee1210669@ee.iitd.ac.in}{ee1210669@ee.iitd.ac.in} & 0.98\\ 
\hline 
\href{Vishal-495}{Vishal Sai Bingi} & 2021EE10668 & \href{mailto:ee1210668@ee.iitd.ac.in}{ee1210668@ee.iitd.ac.in} & 0.7\\ 
\hline 
\href{https://www.linkedin.com/in/chetan-chaurasia-561b3b228}{Chetan Chaurasia} & 2021EE10147 & \href{mailto:ee1210147@ee.iitd.ac.in}{ee1210147@ee.iitd.ac.in} & 0.7\\ 
\hline 
\href{nan}{Chirag Gautam} & 2021EE10166 & \href{mailto:ee1210166@ee.iitd.ac.in}{ee1210166@ee.iitd.ac.in} & 0.7\\ 
\hline 
\href{nan}{Deepanshu Kumar} & 2021EE10696 & \href{mailto:ee1210696@ee.iitd.ac.in}{ee1210696@ee.iitd.ac.in} & 0.4\\ 
\hline 
\href{nan}{Garvit Dhoot} & 2021EE30823 & \href{mailto:ee3210823@ee.iitd.ac.in}{ee3210823@ee.iitd.ac.in} & 0.83\\ 
\hline 
\href{https://www.linkedin.com/in/himanshu-prajapati-400669217}{Himanshu} & 2021EE30177 & \href{mailto:ee3210177@ee.iitd.ac.in}{ee3210177@ee.iitd.ac.in} & 0.7\\ 
\hline 
\href{https://github.com/kalu693}{Kalu Ram Tard} & 2021EE10680 & \href{mailto:ee1210680@ee.iitd.ac.in}{ee1210680@ee.iitd.ac.in} & 0.7\\ 
\hline 
\href{nan}{Kartavya Khurana} & 2021EE30710 & \href{mailto:ee3210710@ee.iitd.ac.in}{ee3210710@ee.iitd.ac.in} & 0.98\\ 
\hline 
\href{BoredApe07}{Pramukh Jain} & 2021EE10720 & \href{mailto:ee1210720@ee.iitd.ac.in}{ee1210720@ee.iitd.ac.in} & 0.98\\ 
\hline 
\href{https://github.com/raviparihar0659}{Ravi Parihar} & 2021EE10156 & \href{mailto:ee1210156@ee.iitd.ac.in}{ee1210156@ee.iitd.ac.in} & 0.82\\ 
\hline 
\href{https://www.linkedin.com/in/sheetal-manatawal-50119a236}{Sheetal Manatawal} & 2021EE10174 & \href{mailto:ee1210174@ee.iitd.ac.in}{ee1210174@ee.iitd.ac.in} & 0.7\\ 
\hline 
\href{https://github.com/shivam-kumar04}{Shivam Kumar} & 2021EE10165 & \href{mailto:ee1210165@ee.iitd.ac.in}{ee1210165@ee.iitd.ac.in} & 0.7\\ 
\hline 
\href{nan}{Sudhanshu Raj} & 2021EE10132 & \href{mailto:ee1210132@ee.iitd.ac.in}{ee1210132@ee.iitd.ac.in} & 0.0\\ 
\hline 
\href{https://github.com/tanmaimerugu}{Tanmai Merugu} & 2021EE10149 & \href{mailto:ee1210149@ee.iitd.ac.in}{ee1210149@ee.iitd.ac.in} & 0.7\\ 
\hline 
\href{https://github.com/vikas4vikas}{Vikas Meena} & 2021EE10169 & \href{mailto:ee1210169@ee.iitd.ac.in}{ee1210169@ee.iitd.ac.in} & 0.7\\ 
\hline 
\href{nan}{Vinay Sah} & 2021EE10171 & \href{mailto:ee1210171@ee.iitd.ac.in}{ee1210171@ee.iitd.ac.in} & 0.7\\ 
\hline 
\hline
		    \caption{IF less than 1 (Others)}
	    \end{longtable}
    \end{center}
{\tiny \textcolor{white}{\ac{IF}}}


\listoftables
\addcontentsline{toc}{section}{List of Tables}
\listoffigures
\addcontentsline{toc}{section}{List of Figures}

\newpage
\section{Acronyms}
%adding the acromyns
\input{Files/Chapters/Introduction/acronym.tex}

%\newpage

\section{Mind Maps}
\begin{center}
    \begin{figure}[H]
        \centering
        \includegraphics[width=0.9\textwidth,trim={1cm 0cm 0cm 2cm}, clip]{./Files/Images/MinpMap.jpg}
        \caption{\textbf{Mind Map emphasizing the requirements} (Made using \textit{MindNote})}
    \end{figure}
    
%\newpage

\begin{landscapes}
    \section{Project Management Details}

    
    \begin{figure}[H]
        \centering
        \includegraphics[width=\textwidth, trim={0cm 6cm 0cm 0.5cm}, clip]{../../Images/Network.pdf}
        \caption{\textbf{Network Chart} {(Created using \textit{Project Libre})}}
    \end{figure}
    % \footnotetext{Red boxes denote the critical path}
    
    \newpage
    
    \begin{figure}[H]
        \centering
        \includegraphics[width=\textwidth]{../../Images/WBS.pdf}
        \caption{\textbf{\ac{WBS} Chart} {(Created using \textit{Project Libre})}}
    \end{figure}

\end{landscapes}


\begin{figure}[H]
	\begin{subfigure}[b]{\textwidth}
		\centering
		\includegraphics[page=1,width=\textwidth,trim={0cm 0cm 0cm 0cm}, clip]{./Files/Images/Resources.pdf}
		\caption{}
	\end{subfigure}
	\caption[\textbf{Resource Breakdown}]{\textbf{Resource Breakdown} {(Created using \textit{Project Libre})}}
\end{figure}

\newpage

\begin{figure}[H]
	\continuedfloat
	\begin{subfigure}[b]{\textwidth}
		\centering
		\includegraphics[page=2,width=\textwidth]{./Files/Images/Resources.pdf}
		\caption{}
	\end{subfigure}
	\caption[\textbf{Resource Breakdown}]{\textbf{Resource Breakdown} {(Created using \textit{Project Libre})}}
\end{figure}
\pagebreak



\section{Abstract}

his project introduces a system to assist users availing the bus facility inside the campus. Our solution consists of a Bus Mounted Unit that shows information about the bus's status on each bus as well as a sturdy, easily readable Display Unit at each stop that informs passengers of the bus's anticipated arrival time (as well as other important details). In addition to making an effort to be economical and energy-efficient, the system helps passengers by giving them important and beneficial details about the buses. This paper details the design process and technological integrations of this comprehensive bus assistance solution.


\pagenumbering{arabic}

\section{Motivation}

The existing bus transit system on campus lacks effectiveness, accessibility, and convenience. Uncertain wait times make it challenging for passengers to decide whether to wait for the next bus or seek alternative transportation. Recognizing this limitation, we are developing a solution to address this issue and enhance the convenience of the bus transportation system for passengers. We maximise renewable energy (in the form of solar energy) usage for this solution to minimise any overhead power costs and make a greener solution.




\pagebreak

%%%%%%%%%% Requirements %%%%%%%%%%%%%%%%


\chapter{Requirements}
\section{Functional Requirements}

\begin{enumerate}
    \item \underline{\textsc{Passenger Wait Times}} : Speeding to compensate for delays, even after a passenger waits over a minute and notifies the driver, is prohibited. Safety is paramount and arrival times may fluctuate accordingly.

    \item \underline{\textsc{Driver Notification}} : There's no obligation to inform the driver if a passenger waits for more than a minute. Waiting time is restricted to passenger boarding and deboarding.

    \item \underline{\textsc{Bus Full Indicator}} : The driver is not obligated to provide information to waiting passengers regarding the availability of space.

    \item \underline{\textsc{Road Incidents Update}} : When there's an unexpected delay or a roadblock, the display will flash '999' to signal an emergency.

    \item \index{Feedback Mechanism}\underline{\textsc{Feedback Mechanism}} : A digital interface could be implemented for users to provide feedback or report issues related to bus services.

    \item \index{Automation}\underline{\textsc{Automation of Bus Unit Operation}} :  It needs to be fully automated, manual input from the driver or staff for every bus direction change or stop is not expected.

    \item \underline{\textsc{Wait Time Interface}} : An interface can be devised to allow individuals heading to the bus stop to access information regarding the wait time.

    \item \underline{\textsc{Passenger Notification}} : Notifying the bus driver about a waiting passenger does not affect the bus route, even if the bus is empty.

    \item \underline{\textsc{Empty Bus Stops}} : The bus stops at empty bus stop as well, for passenger deboarding.

    \item \underline{\textsc{System Timeline}} : The desired timeline for the development, testing, and deployment of the system is April 25th, 2025.

    \item \index{Real-Time Tracking}\underline{\textsc{Real-Time Tracking}} : Real-time tracking of every bus is not required, but it is suggested to keep track of location and travel activity logs for around a month. Only have to keep a log of arrival and departure times at every bus stop, and this activity log is to be stored for the long term. (Can be done at appropriate frequency, no need for real time)

    \item \index{U-Turn}\underline{\textsc{Bus \gls{U-turn}}} : There may not be sufficient space at every stop for the bus to make a \gls{U-turn} and onboard passengers on the reverse path. Bus follows its route strictly, does not make U-Turns.

    \item \index{Accessibility}\underline{\textsc{Accessibility Priority}} : Prioritizing full accessibility for all campus residents on the bus service is not necessary.

    \item \underline{\textsc{Location Exchange}} : There's no need for each bus to be informed about the location of other buses.

    \item \underline{\textsc{Early Departure}} : Buses should not depart early, even if passengers have been waiting for over a minute, as it can confuse passengers and cause inconvenience.

    \item \underline{\textsc{Stop Wait Time}} : Buses typically aim to depart from each stop promptly after passengers have finished boarding and alighting.

    \item \underline{\textsc{Arrival and Departure Logging}} : The logging of bus arrival and departure must be automatic and cannot rely on manual input, ensuring accuracy and efficiency in tracking bus movements.

    \item \underline{\textsc{Centralized Management}} : The bus service is centrally managed, with everything logged. While there's no real-time monitoring, logs are available for review.

    \item \underline{\textsc{Passenger Drop-off Notification}} : There is no notification received when a passenger enters the bus regarding their drop-off.

    \item \underline{\textsc{Bus Bypass Protocol}} : If the bus is already full, the driver does not need to stop the bus at every subsequent stop.

    \item \underline{\textsc{Connectivity Focus}} :Connectivity is exclusively between buses and stops, with no direct communication link to passengers.

    \item \underline{\textsc{Capacity Information}} : It is unnecessary to display a bus's full status at upcoming stops, as passengers may alight at any point, which remains unpredictable to the driver.

    \item \underline{\textsc{Emergency Stops}} : In emergency situations, like a passenger feeling nauseous, the bus will make unscheduled stops between designated stops to address the issue and prioritizing passenger safety and well-being.

    \item \underline{\textsc{Waiting Passengers}} : The system should distinguish between transient presence and waiting passengers standing up to 5 feet from the bus stop pole for more than 30 seconds. Any passerby standing for 0.5 minutes would qualify them as a potential passenger, this wrong identification is allowed.

    \item \underline{\textsc{Following Route}} : Route to be followed and stops to be made should be a fully automated decision.

    \item \underline{\textsc{Location of Waiting Passengers}} : The passenger can be waiting anywhere along the route, not necessarily at the next bus stop.

    \item \underline{\textsc{Account for Delays}}: Assuming the bus may not always reach X at 8 AM sharp to start the shift.
\end{enumerate}

% \begin{enumerate}
%     \item \underline{\textsc{Passenger Wait Times and Safe Bus Speeds}} : Speeding to compensate for delays, even after a passenger waits over a minute and notifies the driver, is prohibited. Safety is paramount and arrival times may fluctuate accordingly.

%     \item \underline{\textsc{Updating Timings for Road Incidents}} : When there's an unexpected delay or a roadblock, the display will flash '999' to signal an emergency.

%     \item \underline{\textsc{Automation of Bus Operations}} :  It needs to be fully automated, manual input from the driver or staff for every bus direction change or stop is not expected.

%     \item \underline{\textsc{Bus Route and Passenger Notification}} : Notifying the bus driver about a waiting passenger does not affect the bus route, even if the bus is empty.

%     \item \underline{\textsc{Stopping at Empty Bus Stops}} : The bus is not expected to make stops at empty bus stops.

%     \item \underline{\textsc{Real-Time Bus Tracking}} : Real-time tracking of every bus is not required, but it is suggested to keep track of location and travel activity logs for around a month.

%     \item \underline{\textsc{Bus U-Turn for Passengers}} : There may not be sufficient space at every stop for the bus to make a U-turn and onboard passengers on the reverse path.

%     \item \underline{\textsc{Early Bus Departure}} : Buses should not depart early, even if passengers have been waiting for over a minute, as it can confuse passengers and cause inconvenience.

%     \item \underline{\textsc{Bus Stop Wait Time}} : Buses typically aim to depart from each stop promptly after passengers have finished boarding and alighting.

%     \item \underline{\textsc{Bus Arrival and Departure Logging}} : The logging of bus arrival and departure must be automatic and cannot rely on manual input, ensuring accuracy and efficiency in tracking bus movements.

%     \item \underline{\textsc{Centralized Bus Service Management}} : The bus service is centrally managed, with everything logged. While there's no real-time monitoring, logs are available for review.

%     \item \underline{\textsc{Passenger Drop-off Notification}} : There is no notification received when a passenger enters the bus regarding their drop-off.

%     \item \underline{\textsc{Bus Bypass Protocol}} : If the bus is already full, the driver does not need to stop the bus at every subsequent stop.

%     \item \underline{\textsc{Emergency Stops Policy}} : In emergency situations, like a passenger feeling nauseous, the bus will make unscheduled stops between designated stops to address the issue and prioritizing passenger safety and well-being.

%     \item \underline{\textsc{Differentiation of Waiting Passengers}} : The system should distinguish between transient presence and waiting passengers. Standing for 0.5 minutes qualifies someone as a potential passenger,with false positives accepted.
% \end{enumerate}

\section{Logistical Requirements}

\begin{enumerate}

\item \index{Stop Limits}\underline{\textsc{Stop Limits}} : The number of stops allowed between origin and destination is predetermined by the institute's system.

\item \underline{\textsc{Stop Distribution}} : The spacing between stops is not consistent and needs to be surveyed for accurate measurement.

\item \underline{\textsc{Bus Operation}} : If the number of passengers at the starting station exceeds the capacity of one bus, two buses can be operated simultaneously.

\item \underline{\textsc{Bus Deployment}} : At the moment, there are only 2 buses available. The design needs to consider scalability aspects to accommodate demand fluctuations.

\item \underline{\textsc{Bus Speed}} : Bus speed can vary depending on the amount of traffic, the number of passenger stops, the time of day, the weather, and the quality of the road.

\item \underline{\textsc{Alternative Transportation}} : No alternative transportation options are readily accessible to users.

\item \index{Design Scalability}\underline{\textsc{Design Scalability}} : When designing the architecture, the primary scalability considerations to focus on are the number of buses and the number of stations.
\end{enumerate}
\section{Interface Requirements}

\begin{enumerate}
    \item \index{Display Method}\underline{\textsc{Display Method}} : Use of an \index{LED dot matrix display}LED dot matrix display.

    \item \underline{\textsc{Accuracy}} :The reading on the display must be accurate to ±1 minute.

    \item \underline{\textsc{Duration Of Activity}} : Must remain active only between 0700 to 1930 hours, powering on/off should be automated.

    \item \index{Display Mounting}\underline{\textsc{Display Mounting}} : Secure mounting to prevent theft. Mechanical fixation at unreachable height.

    \item \underline{\textsc{Accessibility for Visually Impaired}} :  Utilize PWM output or speech IC for accessibility.

    \item \index{Syncing Capacity}\underline{\textsc{Syncing Capacity}} : Not necessary for synchronization.

    \item \index{Budget Constraints}\underline{\textsc{Budget Constraints}} : To be Made as cost-efficient as possible.

    \item \index{Passenger Input}\underline{\textsc{Passenger Input}} : Prefer sensor-based input.

    \item \underline{\textsc{Audio Announcement}} : Consider installing an audio unit.

    \item \index{BUk Functionality}\underline{\textsc{BUk Functionality}} : No display on BUk, includes some visual confirmation LEDs.

    \item \underline{\textsc{Passenger Waiting Notification}} : Separate indicator integrated into BUk.

    \item \index{Error Indicator}\underline{\textsc{Error Indicator}} : The Display would read ‘999’ to indicate emergency or unforeseen delays

    \item \underline{\textsc{Feedback System}} : Are there any means for taking feedback from the users or for reporting issues related to bus services?

    \item \underline{\textsc{Resilient}} : Display is exposed to all Delhi weather conditions.
   
\end{enumerate}
\section{Readability Requirements}

\begin{enumerate}
    \item \index{Base Visibility}\underline{\textsc{Base Visibility}} : Must be readable from any point by a person with \index{6/6 vision}6/6 vision standing at upto 8 feet in front of the bus stop.

    \item \underline{\textsc{Visibility in Varied Weather}} : Clarification on visibility criteria under different weather conditions. Consideration of weather factors in New Delhi, including smog and rain.

    \item \underline{\textsc{Performance Metrics}} : Identification of specific performance metrics. Proposal to use start times, stop arrival/departure times, and out-of-service durations for evaluation.

\end{enumerate}
\section{Power Requirements}

\begin{enumerate}
    \item \index{Power Supply}\underline{\textsc{Power Supply}} : Display Unit must be battery-powered. Solar power to be used. It cannot be delivered directly to the embedded system.

    \item \underline{\textsc{Alternative Power}} : Solar energy reliability in peak Delhi winter. Suggestions for backup power sources such as larger batteries or panels. Consensus needed on procurement across stakeholders.

    \item \underline{\textsc{Connection Details}} : \\Solar Panel$\to$ AC/DC converter $\to$ capacitor $\to$ Embedded system \\ \makebox[\linewidth]{OR}\\
Solar Panel $\to$ \index{PWM converter}PWM converter (or \index{MPPT converter}MPPT converter) $\to$ USB $\to$ \index{Embedded Systems}Embedded Systems

    \item \underline{\textsc{Location of Source of Solar Power}} :Must be installed at a nearby sunny point and directed to nearby bus stop
\end{enumerate}
\section{Maintenance Requirements}

\begin{enumerate}
    \item \underline{\textsc{Maintenance Availability and Feasibility}} : Human resources are available for daily maintenance of the stops, making monthly maintenance feasible.

    \item \underline{\textsc{Human Support for Technical Issues}} : While the system is designed for minimal human intervention, technical support is available on an as-needed basis to address any technical issues that may arise.

    \item \underline{\textsc{Bus Type and Out-of-Service Status}} : Bus type (electric or not) is not relevant .bus will be labeled as "Out of Service" during refueling, washing, and other maintenance issues. Maintenance checks to be carried out on a daily basis. 

    \item \underline{\textsc{Stall Period for Display Unit Malfunction}} : In the event of an unforeseen malfunction in the display unit, a stall period of 1 hour will be allowed to fix the issue.
\end{enumerate}



%%%%%%%%%% Specifications %%%%%%%%%%%%%%%%


\chapter{Specifications}
\section{\index{Backend Architecture}Backend Architecture Specifications}

\begin{itemize}
    \item There are $n$ buses: $0, 1, 2, \ldots n-1$
    \item There are $m$ stops: $0, 1, 2, \ldots m-1$  ($0$: first stop, $m-1$: last stop)
    \item We have to synchronize the \textit{\gls{24-hour clock}} at each stop
    \item In this system, all bus shifts end at the starting stop, i.e. the bus shift ends when the bus is back at the $0$\textsuperscript{th} stop. The sequence followed in a shift is \\
          {\bfseries $0$ $\to$ $1$ $\to$ $2$ $\to$ $3$ $\to$ \ldots $\to$ $m-2$ $\to$ $m-1$ $\to$ $m-2$ $\to$ \ldots$\to$ $3$ $\to$ $2$ $\to$ $1$ $\to$ $0$}\\
\end{itemize}

\begin{figure}[H]
        \centering
        \includegraphics[width=0.7\textwidth]
        {./Files/Images/Flowchart.png}
        \caption{\textbf{Example route for m = 4}} % Add a caption
    \end{figure}

\begin{itemize}
    \item All these stops will be connected to the institute’s WiFi service, which will be used to transmit relevant information to all the stops 
    \item⁠ We use WiFi even when the ESP32 has the option of BLE because:
\begin{itemize}
    \item BLEs indoor range is around 100 m, but outdoors the signals are attenuated and therefore the range reduces
    \item We need to cover a route of approximately 2 km using 4 bus units. Therefore using BLE is not feasible unless we can have bus stops every 50-100 metres, which seems impractical.
\end{itemize}
\end{itemize}
\subsection{Data Storage}
\subsubsection{Over the Network}

\begin{enumerate}

\item Maintain the \textbf{Arrival}\textsubscript{mxn} matrix : Element $(i, j)$ stores the arrival time of $j$\textsuperscript{th} bus at the $i$\textsuperscript{th} stop.
\item Maintain the \textbf{Departure}\textsubscript{mxn} matrix : Element $(i, j)$ stores the departure time of $j$\textsuperscript{th} bus at the $i$\textsuperscript{th} stop.
\item Maintain the \textbf{PassengerWaitingAtStop}\textsubscript{1xm} matrix : Element $(0, j)$ is $x$ where
 \[ x = \begin{cases} \mbox{1} & \mbox{if any passenger is waiting at the j\textsuperscript{th} stop}  \\ \mbox{0} & \mbox{otherwise} \end{cases} \]
\item Maintain the \textbf{Direction}\textsubscript{1xn} matrix : Element (0, j) is x where
 \[ x = \begin{cases} \mbox{0} & \mbox{if $j$\textsuperscript{th} bus is moving towards $0$\textsuperscript{th} stop}  \\ \mbox{1} & \mbox{if $j$\textsuperscript{th} bus is moving towards $m-1$\textsuperscript{th} stop} \end{cases} \]


\end{enumerate}
\subsubsection{Locally}

\begin{enumerate}

\item Maintain the \textbf{Scheduled}\textsubscript{mxn} matrix: Element $(i, j)$ stores the scheduled time of the $j$\textsuperscript{th} bus at the $i$\textsuperscript{th} stop.
\item Maintain the \textbf{Estimated}\textsubscript{mxn} matrix: Element $(i, j)$ stores the estimated time of the $j$\textsuperscript{th} bus at the $i$\textsuperscript{th} stop.
\item These matrices can be stored for estimation of arrival times, the passenger waiting functionality.
\item When $j$\textsuperscript{th} bus arrives at the $i$\textsuperscript{th} stop: $(i, j)$\textsuperscript{th} element of the \textbf{Arrival} matrix is updated. Bus stop `i' acts as the server, and all other bus stops act as clients.
\item When j\textsuperscript{th} bus departs the i\textsuperscript{th} stop: $(i, j)$\textsuperscript{th} element of the \textbf{Departure} matrix is updated. All arrival times of $j$\textsuperscript{th} column are updated to new estimate times based on \textit{average} time between stops and departure time from $i$\textsuperscript{th} stop.
\item For sensing whether the bus has stopped or not, we are using \textit{Accelerometer} reading.
\item \textbf{Bus stop nodes that can access this matrix} (i.e. nodes that are connected to the network):
\begin{itemize}
\item Arrival time of next bus at $i$\textsuperscript{th} stop= min\{$i$\textsuperscript{th} row\}
\item Suppose that $j$ the next bus arriving at $i$\textsuperscript{th} stop $= j$.
\item Direction of next bus $= j$\textsuperscript{th} element of \textbf{Direction} matrix.
\item Both Arrival time and Direction of buses needs to be communicated to the commuters.
\end{itemize}
\item \textbf{Bus stop nodes that cannot access this matrix} (i.e. nodes that unexpectedly get disconnected):
\begin{itemize}
\item Arrival times of each bus estimated using scheduled time and last known whereabouts of buses:\\
Estimated time = Scheduled time + (last known deviation from scheduled 
time)
\item If estimated time shows a large deviation from scheduled time, we will use scheduled time instead of estimated time.
\item Direction matrix can be updated based on these estimates: Direction of $j$\textsuperscript{th} bus switches when arrival time of stop $m-1$ is crossed.
\item Similar to the previous case, both Arrival time and Direction of buses needs to be communicated to the commuters.
\end{itemize}
\item \textbf{When the bus reaches stop $0$ after a round trip}:
\begin{itemize}
\item  The \textbf{Arrival} and \textbf{Departure} matrices are updated to actual/estimated arrival and departure values, and this can be used to keep log of actual arrival and departure times in the system.
\item Moving average of previous runs can be used to refine predictions.
\item  After uploading these times, we reset \textbf{Arrival} and \textbf{Departure} matrices to scheduled times.
\end{itemize}
\item \textbf{For more frequent time updates:}
\begin{itemize}
\item We can specify some locations (using \textit{GPS}) where Wi-Fi is available on the route and send quick location updates to the network.
\item These locations can be used as \textit{pseudo-stops}, where the Bus Unit directly informs all Bus Stop nodes its location which can be further used to refine time predictions.
\end{itemize}
\end{enumerate}
\section{Frontend Specifications}
\subsection{Display Unit}
\begin{enumerate}
    \item These units will be mounted at every bus stop
    \item This unit will tell the time left for the buses from both direction to arrive at the station using the \gls{sliding text animation}
    \item There will be one display unit per bus stop. Directional information will be displayed on the unit for buses from both directions
    \item When the units cannot connect to network, the time of arrival will be an extrapolation of the past arrival time data. The time of arrival will blink telling the maintenance staff that the unit is not connected to the network, but for the normal people it will be just the time of arrival
    \item The display unit will be solar power. This is the connection to be followed: \\Solar Panel$\to$ AC/DC converter $\to$ capacitor $\to$ Embedded system \\ \makebox[\linewidth]{OR}\\
          Solar Panel $\to$ \index{PWM converter}\gls{PWM converter} (or \index{MPPT converter}\gls{MPPT converter}) $\to$ \ac{USB} $\to$ \index{Embedded Systems}Embedded Systems
\end{enumerate}
\subsection{Bus Mounted Units}
\begin{enumerate}
    \item Every bus will have a unit (BUk for the $k^{th}$ bus)
    \item The units will be powered from the battery of the bus
    \item While the bus is moving, these units will never be turned OFF
    \item These units will contain \gls{accelerometer} (to determine the state of the bus whether it is moving or not)
    \item These units will also contain a RED switch, when pressed it will signal the system that the bus is OUT OF SERVICE
\end{enumerate}

\newpage
\section{Cost Specifications}

\subsection{Bus Stop Unit}
The individual components are as follows:


\begin{table}[h]
    \centering
    \begin{tabular}{|c|c|c|}
    \hline
    \textbf{Component} & \textbf{Cost} & \textbf{Source} \\
    \hline \hline
         Solar Charge Controller & \textbf{Rs. 150} & \cite{prabha}\\
         \hline
        10W/12V Solar Panel & \textbf{Rs. 890} & \cite{premium} \\
        \hline
    ESP32 Development Board with Wifi and Bluetooth & \textbf{Rs. 345} & \cite{esp32} \\
    \hline
    Ultrasonic Distance Sensor Module - HC-SR04 & \textbf{Rs. 49} & \cite{ultrasonic} \\
    \hline
    2 Way Rectangular Sintex Pole Junction Box &  \textbf{Rs. 270} & \cite{sintex} \\
    \hline
    MAX7219 8\texttimes8 LED Dot Matrix Display Module \texttimes \ 3 & \textbf{Rs. 119} & \cite{max7219} \\
    \hline
    5mm LED - White Color & \textbf{Rs. 2} & \cite{a5mm}\\
    \hline
    \end{tabular}
    \caption{Bus Stop Unit Components}
    \label{tab:my_label}
\end{table}
    
Total Cost for each Bus Stop Unit: Sum of individual components = \textbf{Rs. 2061}
\subsection{Bus Unit}
The individual components are as follows
\begin{table}[h]
    \centering
    \begin{tabular}{|c|c|c|}
    \hline
    \textbf{Component} & \textbf{Cost} & \textbf{Source} \\
    \hline \hline
    ESP32 Development Board with Wifi and Bluetooth & \textbf{ Rs. 345} & \cite{esp32} \\
    \hline
    5mm Bi-color LED Red Green 3 Pin - Common Anode & \textbf{Rs. 3} & \cite{max7219} \\
    \hline
    5mm LED - White Color & \textbf{Rs. 2} & \cite{a5mm} \\
    \hline
    MPU6050 Gyroscope/Accelerometer Sensor & \textbf{Rs. 99} & \cite{MPU6050}\\
    \hline
    \end{tabular}
    \caption{Bus Stop Unit Components}
    \label{tab:my_label}
\end{table}

    

Total Cost for each Bus Unit: Sum of individual components = \textbf{Rs. 449}
\newpage

\section{Manpower Specifications}

\subsection{Man Hours}
\begin{center}
    \label{table:man_hours}
    \begin{longtable}{ | c | e | c | c| }
        \hline
        \multirow{2}{*}{\textbf{Name}} & \multirow{2}{*}{\textbf{Entry Number}} & \multirow{2}{*}{\textbf{Subtribe}} & \textbf{Man Hours} \\
                                       &                                        &                                    & \textbf{(in hrs)}  \\
        \hline \hline
        %%%
        Navneet Raj                    & 2021MT10240                            & Documentation                      & 10.0               \\
        \hline
        Sanjay Pooniya                 & 2021EE10148                            & Documentation                      & 3.0                \\
        \hline
        Ayush Gupta                    & 2021MT10697                            & Documentation                      & 4.0                \\
        \hline
        Rishabh Barola                 & 2021EE10636                            & Documentation                      & 5.0                \\
        \hline
        Asmit Singh                    & 2021MT10887                            & Documentation                      & 5.0                \\
        \hline
        Vasu Sharma                    & 2021EE10620                            & Documentation                      & 9.0                \\
        \hline
        Harsh Agarwal                  & 2021EE30977                            & Others                             & 4.0                \\
        \hline
        Abhinav Verma                  & 2021EE10978                            & Others                             & 10.0               \\
        \hline
        Khushika Shringi               & 2021EE10665                            & Others                             & 10.0               \\
        \hline
        Yash Goel                      & 2021EE10984                            & Others                             & 9.0                \\
        \hline
        Aryan Gupta                    & 2021EE10974                            & Others                             & 9.0                \\
        \hline
        Aniket Singh                   & 2021MT10256                            & Others                             & 3.0                \\
        \hline
        2021ee10653                    & 2021EE10653                            & Others                             & 5.0                \\
        \hline
        Mridul Ahi                     & 2021MT10901                            & Others                             & 1.0                \\
        \hline
        Eepsita                        & 2021EE10692                            & Others                             & 7.0                \\
        \hline
        Pooja Mahajan                  & 2021EE10652                            & Others                             & 9.0                \\
        \hline
        Chetan Chaurasia               & 2021EE10147                            & Others                             & 6.0                \\
        \hline
        Vikas Meena                    & 2021EE10169                            & Others                             & 4.0                \\
        \hline
        Anshul                         & 2021EE10729                            & Others                             & 5.0                \\
        \hline
        Ark Verma                      & 2021EE10783                            & Others                             & 10.0               \\
        \hline
        Yash Agarwal                   & 2021EE10638                            & Others                             & 8.0                \\
        \hline
        Sheetal Manatawal              & 2021EE10174                            & Others                             & 5.0                \\
        \hline
        Disha Katia                    & 2021EE10647                            & Others                             & 6.5                \\
        \hline
        Chirag Gautam                  & 2021EE10166                            & Others                             & 4.0                \\
        \hline
        Shubham Aggarwal               & 2021EE10809                            & Others                             & 8.0                \\
        \hline
        Mohit Raj                      & 2021MT10919                            & Others                             & 5.0                \\
        \hline
        Ujjwal Yadav                   & 2021EE10669                            & Others                             & 8.0                \\
        \hline
        Aniket Abhiraj                 & 2021EE10676                            & Others                             & 9.0                \\
        \hline
        Sarthak Kumar Singh            & 2021EE10673                            & Others                             & 8.0                \\
        \hline
        Aditi Shekhar                  & 2021EE10685                            & Others                             & 6.0                \\
        \hline
        Anchal                         & 2021MT10910                            & Others                             & 8.0                \\
        \hline
        Rohan Das                      & 2021EE10621                            & Others                             & 7.0                \\
        \hline
        Rohan Das                      & 2021EE10621                            & Others                             & 7.0                \\
        \hline
        Harsh Swaika                   & 2021EE11052                            & Others                             & 4.0                \\
        \hline
        Aditya Jain                    & 2021EE10633                            & Others                             & 7.0                \\
        \hline
        Harshit Sachdeva               & 2021EE30705                            & Others                             & 7.0                \\
        \hline
        Kinjal Anchhara                & 2021MT60959                            & Others                             & 8.0                \\
        \hline
        Dhruv Kushwaha                 & 2021MT10235                            & Others                             & 6.0                \\
        \hline
        Shubh Chhabra                  & 2021EE10645                            & Others                             & 8.0                \\
        \hline
        Pramukh Jain                   & 2021EE10720                            & Others                             & 6.0                \\
        \hline
        Oshin Kavdia                   & 2021EE10654                            & Others                             & 4.0                \\
        \hline
        Kalu Ram Tard                  & 2021EE10680                            & Others                             & 8.0                \\
        \hline
        Joel Arun Kumar Yenubotula     & 2021EE10159                            & Others                             & 5.0                \\
        \hline
        Ameya Mishra                   & 2021MT10637                            & Others                             & 9.0                \\
        \hline
        Tanmai Merugu                  & 2021EE10149                            & Others                             & 8.0                \\
        \hline
        Khushvind Maurya               & 2021MT10238                            & Others                             & 3.5                \\
        \hline
        Vinay Sah                      & 2021EE10171                            & Others                             & 3.0                \\
        \hline
        Shivam Kumar                   & 2021EE10165                            & Others                             & 4.0                \\
        \hline
        Deepanshu Kumar                & 2021EE10696                            & Others                             & 4.0                \\
        \hline
        Durgesh Nandini                & 2021EE10651                            & Others                             & 8.0                \\
        \hline
        Kaustubh Dev                   & 2021EE10689                            & Others                             & 8.0                \\
        \hline
        Arnav Goel                     & 2021EE10699                            & Others                             & 7.0                \\
        \hline
        Ayush Kumar                    & 2021EE10150                            & Others                             & 5.0                \\
        \hline
        Bingi Sai Vishal               & 2021EE10668                            & Others                             & 6.0                \\
        \hline
        Abhilasa Das                   & 2021EE10168                            & Others                             & 8.0                \\
        \hline
        Ravi Parihar                   & 2021EE10156                            & Others                             & 4.0                \\
        \hline
        Himanshu Prajapati             & 2021EE30177                            & Others                             & 4.0                \\
        \hline


        %%%
        \hline
        \caption{Manpower Specifications}
    \end{longtable}
\end{center}


\subsection{Skillsets Acquired}
\begin{center}
    \label{table:skillsets}
    \begin{longtable}{ | c | c | m{6cm} | }
        \hline
        \textbf{Name}              & \textbf{Subtribe} & \textbf{Skillsets Acquired}                                             \\
        \hline \hline
        %%%
        Navneet Raj                & Documentation     & Overleaf, LaTeX, GitHub Actions, pandoc                                 \\
        \hline
        Sanjay Pooniya             & Documentation     & Latex                                                                   \\
        \hline
        Ayush Gupta                & Documentation     & Research                                                                \\
        \hline
        Rishabh Barola             & Documentation     & RF communication, System design                                         \\
        \hline
        Asmit Singh                & Documentation     & Documentation                                                           \\
        \hline
        Vasu Sharma                & Documentation     & LaTeX, GitHub, Obsidian/Mermaid, MSWord                                 \\
        \hline
        Harsh Agarwal              & Others            & Research                                                                \\
        \hline
        Abhinav Verma              & Others            & Research                                                                \\
        \hline
        Khushika Shringi           & Others            & Research                                                                \\
        \hline
        Yash Goel                  & Others            & Research                                                                \\
        \hline
        Aryan Gupta                & Others            & Research                                                                \\
        \hline
        Aniket Singh               & Others            & Research based work on how to transfer data with minimum internet usage \\
        \hline
        2021ee10653                & Others            & Research                                                                \\
        \hline
        Mridul Ahi                 & Others            & Research                                                                \\
        \hline
        Eepsita                    & Others            & Research                                                                \\
        \hline
        Pooja Mahajan              & Others            & Research                                                                \\
        \hline
        Chetan Chaurasia           & Others            & Research                                                                \\
        \hline
        Vikas Meena                & Others            & Research                                                                \\
        \hline
        Anshul                     & Others            & Research                                                                \\
        \hline
        Ark Verma                  & Others            & RF communication, \ac{GPS} modules, Telemetry, Research                 \\
        \hline
        Yash Agarwal               & Others            & Research                                                                \\
        \hline
        Sheetal Manatawal          & Others            & Research                                                                \\
        \hline
        Disha Katia                & Others            & Research                                                                \\
        \hline
        Chirag Gautam              & Others            & Research                                                                \\
        \hline
        Shubham Aggarwal           & Others            & Research, Problem Solving                                               \\
        \hline
        Mohit Raj                  & Others            & Research                                                                \\
        \hline
        Ujjwal Yadav               & Others            & 7 segment display                                                       \\
        \hline
        Aniket Abhiraj             & Others            & LCD design, Arduino, triger identifier for bus, hardware design.        \\
        \hline
        Sarthak Kumar Singh        & Others            & Research on occupancy estimation                                        \\
        \hline
        Aditi Shekhar              & Others            & Research                                                                \\
        \hline
        Anchal                     & Others            & Research, Browsing, Brainstorming                                       \\
        \hline
        Rohan Das                  & Others            & Research                                                                \\
        \hline
        Rohan Das                  & Others            & Research                                                                \\
        \hline
        Harsh Swaika               & Others            & Research                                                                \\
        \hline
        Aditya Jain                & Others            & Research                                                                \\
        \hline
        Harshit Sachdeva           & Others            & Research                                                                \\
        \hline
        Kinjal Anchhara            & Others            & Research                                                                \\
        \hline
        Dhruv Kushwaha             & Others            & Research, Survey                                                        \\
        \hline
        Shubh Chhabra              & Others            & Research                                                                \\
        \hline
        Pramukh Jain               & Others            & Hardware selection                                                      \\
        \hline
        Oshin Kavdia               & Others            & Research                                                                \\
        \hline
        Kalu Ram Tard              & Others            & Research                                                                \\
        \hline
        Joel Arun Kumar Yenubotula & Others            & ESP32, networks                                                         \\
        \hline
        Ameya Mishra               & Others            & Research                                                                \\
        \hline
        Tanmai Merugu              & Others            & Research                                                                \\
        \hline
        Khushvind Maurya           & Others            & Research                                                                \\
        \hline
        Vinay Sah                  & Others            & Research                                                                \\
        \hline
        Shivam Kumar               & Others            & Research                                                                \\
        \hline
        Deepanshu Kumar            & Others            & Research                                                                \\
        \hline
        Durgesh Nandini            & Others            & Research on LoRa wireless comminication                                 \\
        \hline
        Kaustubh Dev               & Others            & Research, transmission techniques, Receiver hardware                    \\
        \hline
        Arnav Goel                 & Others            & Research                                                                \\
        \hline
        Ayush Kumar                & Others            & Research                                                                \\
        \hline
        Bingi Sai Vishal           & Others            & Analysis and Design of Algorithms, python                               \\
        \hline
        Abhilasa Das               & Others            & Research                                                                \\
        \hline
        Ravi Parihar               & Others            & Research                                                                \\
        \hline
        Himanshu Prajapati         & Others            & GitHub, Research on YT                                                  \\
        \hline


        %%%
        \hline
        \caption{Skillsets Acquired}
    \end{longtable}
\end{center}

\subsection{How Assignment was done}

%The assignment of work for our team project followed a meticulous process to ensure the efficient utilization of each team member's skills and expertise. We initiated the process by \textbf{thoroughly documenting} everyone's \textbf{skillsets}, creating a comprehensive overview of the diverse talents within our team. This documentation played a crucial role in understanding the \textbf{strengths} and \textbf{capabilities} of each individual.

The first step in conducting this assignment efficiently was formation of sub-tribes. Each sub-tribe had 20-25 members. Each of the sub-tribe worked independently of each other to come up with ideas to solve the problem. After a thorough review of all the ideas, one final idea was selected. Approaching the idea formulation by formation of sub-tribes allowed us to come up with multiple great ideas. Each sub-tribe had creative freedom to brainstorm over their ideas. Sub-tribes met regularly over the weekend and within each sub-tribe there was further division of work. This approach resulted in the Tribe being very efficient and creative in its ideas.

Once the idea was finalised, we floated a form to understand the skillsets of the whole team. Based on the responses, the tribe was split into sub-teams to perform specific tasks to ensure the efficient utilization of each team member's skills and expertise. \textbf{Matching} the skills of team members with the demands of each task was a strategic approach that aimed to leverage their \textbf{unique} strengths. This thoughtful allocation of responsibilities not only optimized the utilization of individual skills but also fostered a \textbf{collaborative} environment where team members could complement each other's abilities.


The overall result was a well-coordinated and efficient team, where each member \textbf{contributed meaningfully} to the project. The thorough assignment of tasks according to skill-sets not only enhanced the quality of the work but also facilitated a smoother workflow, ultimately leading to the successful completion of each task within the \textbf{stipulated timelines}.


We used \textbf{OverLeaf} and \textbf{GitHub} to organise the project. The \textbf{.pdf} files were efficiently generate on \textbf{OverLeaf} after any changes to the file. We used \textbf{GitHub} to maintain the versions of the project. 


\subsection{Surplus Manpower}
We \textbf{did not} need any surplus man power.
% Our project's success was underscored by the remarkable synergy and competency within our existing team, obviating the need for surplus manpower. The meticulous process of documenting and aligning individual skillsets with specific tasks ensured that each team member played a crucial and meaningful role. This strategic allocation not only maximized the utilization of available talent but also highlighted the collective strength of our team.



\chapter{Design}


\section{Process Summary}

This is the overall view of the finalized design:
\begin{itemize}

\item \textbf{Initialization: }At the outset, when the bus departs from the depot, the ETA for each bus stop (A and B) will be indicated as 0. Only when the bus arrives at stop A and departs for the first time do bus stop devices generate and exhibit ETAs.

\item
\textbf{Stop A is in contact with Stop B: }
Whenever the bus departs from stop A, ETA is updated at bus stop B; conversely, when the bus returns from stop B to stop A, ETA is also updated at stop A. When a bus stops within a scalable range of any of the bus stops (bus stop units within the scalable range can observe the bus unit), but not in the designated bus stop area, the nearest bus stop unit determines the bus stop's halt time (since the bus unit transmits a signal to the nearest bus stop units). The bus stop unit responsible for detecting the bus's halt promptly notifies the adjacent bus stops, resulting in the freezing of their timers as well, upon detection of the bus's halt. At this particular stopover, the bus may be withdrawn from service.In the event that BUS unit B is not operational, the signal is transmitted to the closest bus stop (where the initial detection of the halt occurred; this signal is subsequently transmitted to the adjacent bus stop, and so forth).

\item
\textbf{Halt outside range: }
In the event that a bus comes to a complete stop in an area beyond the scanning range of any bus stop unit, the timers within the bus stop commence a countdown process until a predetermined threshold is met (or when the bus's expected top arrives within the scannable range of the bus stop).The timers will remain in the frozen state until bus unit B is detected and passes within the scannable range of the bus stop, which is the location where the bus is en route to next.When a passenger is detected by a PIR sensor at a bus stop and the bus is within a visible range of the stop, a notification signal is transmitted to the bus unit that the passengers are awaiting.
\end{itemize}
\section{Network}
\subsection{ESP NOW Protocol}
ESP-NOW is a wireless communication protocol based on the data-link layer, which reduces the five layers of the \gls{OSI model} to only one. This way, the data need not be transmitted through the network layer, the transport layer, the session layer, the presentation layer, and the application layer. Also, there is no need for packet headers or unpackers on each layer, which leads to a quick response reducing the delay caused by packet loss in congested networks.{\tiny \textcolor{white}{\ac{OSI}}}
\begin{figure}[H]
    \centering
    \includegraphics[width=0.75\linewidth]{Files/Images/model-en.png}
    \caption{Comparison of the OSI standard layers and the ESP NOW protocol layers}
    \label{fig:enter-label}
\end{figure}
Here are the key features of the ESP NOW Protocol:
\begin{itemize}
    \item It has a fast and user-friendly pairing method that is suitable for connecting “one-to-many” and “many-to-many” devices, while also controlling them
    \item Occupies fewer CPU and flash resources{\tiny \textcolor{white}{\ac{ECDH}}}
    \item Can be used as an independent protocol that helps with device provisioning, debugging, and firmware upgrades{\tiny \textcolor{white}{\ac{AES}}}
    \item  \gls{ECDH} and \gls{AES}  algorithms make data transmission more secure
    \item The window synchronization mechanism greatly reduces power consumption
\end{itemize}
The devices communicates directly via the use of the data link and pairing do not require the Wi-Fi connection. Pairing can be done by initiating and multiple pairing can be done like one initiator can pair with multiple responder. Using the RSSI, during the paring the device distance can be found and that distance can be authorised and the device at that distance can be paired. The protocol supports long distance communication which will be helpful in using it outdoors. The portocol is compatible with many sensors implying its wide usability to interact among the units or the user and the unit. ESP NOW can also take data logs from multiple responders for analysis. This will facilitate the user to gather the data from the units without needing to go to the unit, again, the units can communicate the same among themselves to correctly estimate the time or find any abnormality in real time. 


\subsection{ESP-WIFI-MESH}
ESP-WIFI-MESH is a networking protocol that operates on top of the Wi-Fi protocol. It facilitates the interconnection of numerous devices, referred to as nodes, across a wide physical area, including both indoor and outdoor spaces, within a single WLAN (Wireless Local-Area Network).It is self-organizing and self-healing meaning the network can be built and maintained autonomously.

\begin{figure}[H]
     \includegraphics[width=0.5\linewidth]{Files/Images/traditional-network.png}
     \includegraphics[width=0.55\linewidth]{Files/Images/esp-wifi-mesh.png}
     \caption{Traditional Wi-Fi Network vs ESP-WIFI-MESH Network}
     \label{fig:enter-label}

\end{figure}

Here are the key features of the ESP-WIFI-MESH:
\begin{itemize}
    \item ESP-WIFI-MESH networks do not rely on a central node like traditional infrastructure Wi-Fi networks. Instead, nodes connect with neighboring nodes, allowing for greater coverage area.
    \item Nodes within this network relay each other's transmissions, enabling interconnectivity without the need to be within range of a central node.
    \item Unlike traditional Wi-Fi networks, ESP-WIFI-MESH is not as susceptible to overloading since the number of nodes permitted on the network is not limited by a single central node's capacity.
\end{itemize}

\subsection{RSSI}
RSSI stands for Received Signal Strength Indicator. It measures how well your device can hear a signal from an access point or router. It’s useful for determining if you have enough signal to get a good wireless connection. Due to the authentication provided by the channel and the incidence of waves from paths of different length, the received signal power is not the same as the transmitted power and RSSI is a measured from the client (receiver).{\tiny \textcolor{white}{\ac{RSSI}}}{\tiny \textcolor{white}{\ac{IEEE}}}\\
RSSI is a term used to measure the relative quality of a received signal to a client device, but has no absolute value. The IEEE 802.11 standard (a big book of documentation for manufacturing Wi-Fi equipment) specifies that RSSI can be on a scale of 0 to up to 255 and that each chipset manufacturer can define their own “RSSI\_Max” value.  \\

\section{Backend}
\begin{center}
    \includegraphics[scale = 0.75]{Files/Images/Img1.jpeg}
\end{center}
We define the following notations : 
\begin{align*}
    T_{AB} &:= A\to B \text{ along path $1$} \\
    T_{BA} &:= B\to A \text{ along path $2$}
\end{align*}
\noindent If the bus stops within the \textsc{BusStopRegion-A}, we update the timer of $A$ to $0$ and update the timer of $B$ to $T_{AB}$ when the bus moves again. If the bus stops within the range of $A$ but outside the bus stop region, we paus the timer for both $A$ and $B$ and resume it when the bus moves again.

\noindent\textbf{Proof}
\begin{center}
    \includegraphics[scale = 0.5]{Files/Images/Img2.jpeg}
\end{center}
We assume that the bus has a fixed acceleration/deceleration profile whenever bus moves and bus moves at $v_{\max}$ from $t_0$ (when bus doesn't stop in between).
\begin{center}
    \includegraphics[scale = 0.5]{Files/Images/Img3.jpeg}
\end{center}

Note that : 
\begin{gather*}
    T_{AB}= t_{a}+t_0 + t_d \\
    T_{AB}' = 2t_{a}+2t_d + t_1+t_2 + T_{\operatorname{stop}} \tag{1}
\end{gather*}

\noindent Since the distance covered is the same, the area under the curves should also be the same. Writing out the expressions for the areas, we get : 
\[\frac12 v_{\max}\cdot T_{AB} = \frac12 v_{\max} (t_a + t_1 + t_d) + \frac12 v_{\max} (t_a+t_2+t_d).\]
Hence, it follows that : 
\[T_{AB} = 2t_a + 2t_d + t_1 + t_2\tag{2}\]
Combining $(1)$ and $(2)$, we get :
\[T_{AB}' = T_{AB} + tT_{\operatorname{stop}}\]
Hence, pausing the timer as soon as the bus is halted, does the job.
\newpage
\section{Components Used}

\subsection{Accelerometer}
The MPU-6050 is a popular integrated circuit that combines a 3-axis gyroscope and a 3-axis accelerometer into a single chip. It's commonly used in various applications, particularly in motion sensing, orientation tracking, and gesture recognition systems.
For precision tracking of both fast and slow motions, the MPU-60X0 features a user-programmable gyroscope full-scale range of ±250, ±500, ±1000, and ±2000°/sec (dps). The parts also have a user-programmable accelerometer full-scale range of ±2g, ±4g, ±8g, and ±16g. The typical operating voltage range is 2.375V to 3.46V with operating current 100µA.
\begin{figure}[H]
    \centering
    \includegraphics[width=1\linewidth]{Files/Images/Accelerometer.jpg}
    \caption{MPU 6050 Details}
    \label{fig:enter-label}
\end{figure}

\subsection{MAX7219/MAX7221}
\subsubsection*{Description}
The MAX7219/MAX7221 are small display drivers with serial input/output capability. They connect microprocessors ($\mu$Ps) to 7-segment numeric LED displays with a maximum of 8 digits, bar-graph displays, or 64 separate displays.

Light-emitting diodes. The integrated circuit contains a BCD code-B decoder, multiplex scan circuitry, segment and digit drivers, and an 8x8 static RAM that saves each digit. A single external resistor is sufficient to establish the segment current for all LEDs. The MAX7221 is compatible with \gls{SPI}™, \gls{QSPI}™, and \gls{MICROWIRE}™ communication protocols. It is equipped with segment drivers that have low slew rates to minimize electromagnetic interference (EMI).
The device features a practical 4-wire serial interface that can be easily connected to any standard microprocessors. Each individual digit can be targeted and modified without having to rewrite the entire display. The MAX7219/MAX7221 additionally provide the user with the option to choose between code-B decoding or no-decode for each digit.
The devices are equipped with a low-power shutdown mode that consumes only $\SI{150}{\mu A}$, as well as analog and digital brightness control. They also have a scan-limit register, which enables the user to display anywhere from 1 to 8 digits. Additionally, there is a test mode available that activates all LEDs simultaneously.{\tiny \textcolor{white}{\ac{SPI}}}{\tiny \textcolor{white}{\ac{QSPI}}}
\subsubsection*{Features}
\begin{itemize}
    \item $\SI{10}{MHz}$ Serial Interface
    \item Individual LED Segment Control
    \item Decode/No-Decode Digit Selection
    \item $\SI{150}{\mu A}$ Low-Power Shutdown (Data Retained)
    \item Digital and Analog Brightness Control
    \item Display Blanked on Power-Up
    \item Drive Common-Cathode LED Display
    \item Slew-Rate Limited Segment Drivers
for Lower EMI (MAX7221)
    \item SPI, QSPI, MICROWIRE Serial Interface (MAX7221)
    \item 24-Pin DIP and SO Packages
\end{itemize}

\begin{figure}[H]
    \centering
    \includegraphics[width=0.7\textwidth]{Files/Images/Max7219.png}
    \caption{Typical Application of \textbf{MAX7219/MAX7221}}
    \label{Max7219}
\end{figure}


\begin{figure}[H]
\centering

\begin{tikzpicture}[node distance=3cm, on grid, auto, thick, initial text=]

    % Define the states
    \node[state, initial] (A) {Counter};
    \node[state, right=of A] (B) {OOS};

    % Draw the transitions
    \path[->]
        (A) edge[loop above] node {X} (A)
        (A) edge[bend left] node {B} (B)
        (B) edge[loop above] node {X} (B)
        (B) edge[bend left] node {A} (A);

\end{tikzpicture}
\caption{FSM diagram for the display}
\end{figure}

%%%%%%%%%%%%%%%%%%%%%%%%%%%%%%%%%%%%%%%%%%%%%
\printnoidxglossaries
%%%%%% Appendices %%%%

\fancyhf{}
\fancyhead[R]{\thepage}
\fancyhead[L]{}

\chapter*{References}
\addcontentsline{toc}{chapter}{References}
\hypertarget{QS}{}

\vbox{\printbibliography[keyword={cite},title={Quoted Sources}]}
\addcontentsline{toc}{section}{Quoted Sources}
\addcontentsline{toc}{section}{Background Research}

\vbox{
  \printbibliography[env=nolabelbib, keyword={no_cite}, title={Background Research}]
  }

\printbibliography[env=nolabelbib, keyword={software}, title={Softwares Used}]

\addcontentsline{toc}{section}{Softwares Used}



\chapter*{Appendices}
\begingroup
%\setcounter{section}{0}
\renewcommand{\thesection}{\Alph{section}}
\addcontentsline{toc}{chapter}{Appendices}
\section*{A. Document Statistics}
\addcontentsline{toc}{section}{A. Document Statistics}
\begin{enumerate}[itemsep = 0.1 pt]

      \item{
            \textbf{Pages:} \zref[abspage]{LastPage}
            }

      % \item{
      %       \textbf{Paragraphs:} 1124
      %       }


      % \item{
      %       \textbf{Lines:} 1471
      %       }

      \item{
            \textbf{Words:} 6969
            }

      \item{
            \textbf{Characters:} 30289
            }
      \item{
            \textbf{Characters (with spaces):} 35898
            }

\end{enumerate}
\newpage
\section*{B. Readability Index}
\addcontentsline{toc}{section}{B. Readability Index}
\renewcommand{\thefootnote}{\alph{footnote}}
\begin{enumerate}[itemsep = 0.1 pt]

      \item{
            \textbf{webfx:} 63.4\footnote{Should be easily understood by \textbf{11} to \textbf{12} year olds.}
            }
      \item{
            \textbf{Flesh-Kincaid Grade Level:} 6.4\footnote{Writing style of the document is comprehensible to individuals who have completed \textbf{seventh} grade.}
            }
      \item{
            \textbf{Flesh-Kincaid Ease Score:} 63.4\footnote{A value between 60 and 80 should be easy for a 12 to 15 year old to understand.}
            }
      \item{
            \textbf{Gunning Fog Index:} 6.8\footnote{The document is written in clear, in fairly understandable language and at a reading level suitable for \textbf{seventh} graders.}
            }
\end{enumerate}
\newpage
\section*{C. Document Information}
\addcontentsline{toc}{section}{C. Document Information}
\renewcommand{\thefootnote}{\alph{footnote}}
\def\mydate{\leavevmode\hbox{\the\year-\twodigits\month-\twodigits\day}}
\def\twodigits#1{\ifnum#1<10 0\fi\the#1}
\begin{enumerate}[itemsep = 0.1 pt]
\item{
\textbf{ID:} P2RS-\mydate-\input{ver}
}
      \item{
            \textbf{Document Type:} Private Release
            }

      \item{
            \textbf{Document Authorised by:} \textbf{\href{mailto:mt1210697@maths.iitd.ac.in}{Ayush Gupta}}, \href{mailto:ee1210137@ee.iitd.ac.in}{\textbf{Aryan Mishra}}}

      \item{
            \textbf{Publication Date:} \today
            }

      \item
            {
            \textbf{Version Number:} \input{ver.tex}\footnote{Tracked automatically in the \textbf{GitHub} Repository.
                  \textbf{v0.0.1} was the first release after P@ release on March 13, 2024.
                  Version number is stepped up, based on whether it is a \textbf{Minor Release} or a \textbf{Bug Fix}.
                  The file after submission at the end of each week is considered a \textbf{Major Release}.}
            }

      \item{\textbf{GitHub Repository Details:} Invite-only \textbf{GitHub Organisation}
            \href{https://github.com/ELP305-Cleaning-Machine}{here}.
            }
\end{enumerate}

\endgroup
%%%%%%%%%%%%%%%%%%%%%%%%%%%%%%
\printindex
\end{document}
