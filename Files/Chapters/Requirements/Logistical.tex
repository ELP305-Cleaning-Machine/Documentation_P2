\section{Logistical Requirements}

\begin{enumerate}

\item \underline{\textsc{Stop Limits on Routes}} : The number of stops allowed between origin and destination is predetermined by the institute's system.

\item \underline{\textsc{Stop Distribution}} : The spacing between stops is not consistent and needs to be surveyed for accurate measurement.

\item \underline{\textsc{Simultaneous Bus Operation}} : If the number of passengers at the starting station exceeds the capacity of one bus, two buses can be operated simultaneously.

\item \underline{\textsc{Bus Deployment Adjustment}} : At the moment, there are only 2 buses available. The design needs to consider scalability aspects to accommodate demand fluctuations.

\item \underline{\textsc{Bus Speed Variability}} : Bus speed can vary depending on the amount of traffic, the number of passenger stops, the time of day, the weather, and the quality of the road.

\item \underline{\textsc{Availability of Alternative Transportation}} : No alternative transportation options are readily accessible to users.

\item \underline{\textsc{Scalability Considerations in Design Architecture}} : When designing the architecture, the primary scalability considerations to focus on are the number of buses and the number of stations.
\end{enumerate}