\subsubsection{Locally}

\begin{enumerate}

    \item Maintain the \textbf{Scheduled}\textsubscript{mxn} matrix: Element $(i, j)$ stores the scheduled time of the $j$\textsuperscript{th} bus at the $i$\textsuperscript{th} stop
    \item Maintain the \textbf{Estimated}\textsubscript{mxn} matrix: Element $(i, j)$ stores the estimated time of the $j$\textsuperscript{th} bus at the $i$\textsuperscript{th} stop
    \item These matrices can be stored for estimation of arrival times, the passenger waiting functionality
\item
    \textbf{Connection of Bus Units with Stop Units:} 
\begin{itemize}
    \item Make the Stop Unit switch to the Bus Unit Wifi when available- this method will use more power
    \item BLE based connection between Bus and Stop Units- this method will be more power efficient but will involve 2 signals on the same antenna, which might lead to unexpected latencies and data losses
\end{itemize}

This mechanism will be finalised only after availability of the hardware, after testing it in our conditions.

    \item For sensing whether the bus has stopped or not, we are using \textit{\gls{accelerometer}} reading.
    \item When $j$\textsuperscript{th} bus arrives at the $i$\textsuperscript{th} stop: $(i, j)$\textsuperscript{th} element of the \textbf{Arrival} matrix is updated. Bus stop `$i$' acts as the server, and all other bus stops act as clients
    \item When $j$\textsuperscript{th} bus departs the $i$\textsuperscript{th} stop: $(i, j)$\textsuperscript{th} element of the \textbf{Departure} matrix is updated. All arrival times of $j$\textsuperscript{th} column are updated to new estimate times based on average time between stops and departure time from $i$\textsuperscript{th} stop
    
    \item \textbf{Bus stop nodes that can access this matrix} (i.e. nodes that are connected to the network):
          \begin{itemize}
              \item Arrival time of next bus at $i$\textsuperscript{th} stop= min\{$i$\textsuperscript{th} row\}
              \item Suppose that $j$ the next bus arriving at $i$\textsuperscript{th} stop $= j$
              \item Direction of next bus $= j$\textsuperscript{th} element of \textbf{Direction} matrix
              \item Both Arrival time and Direction of buses needs to be communicated to the commuters
          \end{itemize}
    \item \textbf{Bus stop nodes that cannot access this matrix} (i.e. nodes that unexpectedly get disconnected):
          \begin{itemize}
              \item Arrival times of each bus estimated using scheduled time and last known whereabouts of buses:\\
                    Estimated time = Scheduled time + (last known deviation from scheduled
                    time)
              \item If estimated time shows a large deviation from scheduled time, we will use scheduled time instead of estimated time
              \item Direction matrix can be updated based on these estimates: Direction of $j$\textsuperscript{th} bus switches when arrival time of stop $m-1$ is crossed
              \item Similar to the previous case, both Arrival time and Direction of buses needs to be communicated to the commuters
          \end{itemize}
    \item \textbf{When the bus reaches stop \textbf{$0$} after a \textit{\gls{round trip}}}:
          \begin{itemize}
              \item  The \textbf{Arrival} and \textbf{Departure} matrices are updated to actual/estimated arrival and departure values, and this can be used to keep log of actual arrival and departure times in the system
              \item Moving average of previous runs can be used to refine predictions
              \item  After uploading these times, we reset \textbf{Arrival} and \textbf{Departure} matrices to scheduled times
          \end{itemize}
\item\textbf{Storing arrival and departure data of buses:}
\begin{itemize}
    \item We have a flash drive in one of the bus units, and all information is ultimately stored in that unit. It can be retrieved when required from the bus unit (manually) by the transport department

    \item We get the permission to upload all this data on transport department’s disk on IITD server.
\end{itemize}

Since the data is not much and also not very important to store over a long period of time, we purge the data every 2 weeks.
These mechanisms will be finalised later, only after analysing the hardware available to us.

          
    \item \textbf{For more frequent time updates:}
          \begin{itemize}
              \item We can specify some locations (using \index{GPS}\textit{\gls{GPS}}) where \textit{\gls{Wi-Fi}} is available on the route and send quick location updates to the network {\tiny \textcolor{white}{\ac{Wi-Fi}}}
              \item These locations can be used as pseudo-stops, where the Bus Unit directly informs all Bus Stop nodes its location which can be further used to refine time predictions
          \end{itemize}
\end{enumerate}